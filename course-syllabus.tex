% Options for packages loaded elsewhere
\PassOptionsToPackage{unicode}{hyperref}
\PassOptionsToPackage{hyphens}{url}
\PassOptionsToPackage{dvipsnames,svgnames,x11names}{xcolor}
%
\documentclass[
  letterpaper,
  DIV=11,
  numbers=noendperiod]{scrartcl}
\usepackage{amsmath,amssymb}
\usepackage{lmodern}
\usepackage{iftex}
\ifPDFTeX
  \usepackage[T1]{fontenc}
  \usepackage[utf8]{inputenc}
  \usepackage{textcomp} % provide euro and other symbols
\else % if luatex or xetex
  \usepackage{unicode-math}
  \defaultfontfeatures{Scale=MatchLowercase}
  \defaultfontfeatures[\rmfamily]{Ligatures=TeX,Scale=1}
\fi
% Use upquote if available, for straight quotes in verbatim environments
\IfFileExists{upquote.sty}{\usepackage{upquote}}{}
\IfFileExists{microtype.sty}{% use microtype if available
  \usepackage[]{microtype}
  \UseMicrotypeSet[protrusion]{basicmath} % disable protrusion for tt fonts
}{}
\makeatletter
\@ifundefined{KOMAClassName}{% if non-KOMA class
  \IfFileExists{parskip.sty}{%
    \usepackage{parskip}
  }{% else
    \setlength{\parindent}{0pt}
    \setlength{\parskip}{6pt plus 2pt minus 1pt}}
}{% if KOMA class
  \KOMAoptions{parskip=half}}
\makeatother
\usepackage{xcolor}
\IfFileExists{xurl.sty}{\usepackage{xurl}}{} % add URL line breaks if available
\IfFileExists{bookmark.sty}{\usepackage{bookmark}}{\usepackage{hyperref}}
\hypersetup{
  pdftitle={Syllabus - STA101L Summer I 2022},
  colorlinks=true,
  linkcolor={blue},
  filecolor={Maroon},
  citecolor={Blue},
  urlcolor={Blue},
  pdfcreator={LaTeX via pandoc}}
\urlstyle{same} % disable monospaced font for URLs
\usepackage{longtable,booktabs,array}
\usepackage{calc} % for calculating minipage widths
% Correct order of tables after \paragraph or \subparagraph
\usepackage{etoolbox}
\makeatletter
\patchcmd\longtable{\par}{\if@noskipsec\mbox{}\fi\par}{}{}
\makeatother
% Allow footnotes in longtable head/foot
\IfFileExists{footnotehyper.sty}{\usepackage{footnotehyper}}{\usepackage{footnote}}
\makesavenoteenv{longtable}
\usepackage{graphicx}
\makeatletter
\def\maxwidth{\ifdim\Gin@nat@width>\linewidth\linewidth\else\Gin@nat@width\fi}
\def\maxheight{\ifdim\Gin@nat@height>\textheight\textheight\else\Gin@nat@height\fi}
\makeatother
% Scale images if necessary, so that they will not overflow the page
% margins by default, and it is still possible to overwrite the defaults
% using explicit options in \includegraphics[width, height, ...]{}
\setkeys{Gin}{width=\maxwidth,height=\maxheight,keepaspectratio}
% Set default figure placement to htbp
\makeatletter
\def\fps@figure{htbp}
\makeatother
\setlength{\emergencystretch}{3em} % prevent overfull lines
\providecommand{\tightlist}{%
  \setlength{\itemsep}{0pt}\setlength{\parskip}{0pt}}
\setcounter{secnumdepth}{-\maxdimen} % remove section numbering
% Make \paragraph and \subparagraph free-standing
\ifx\paragraph\undefined\else
  \let\oldparagraph\paragraph
  \renewcommand{\paragraph}[1]{\oldparagraph{#1}\mbox{}}
\fi
\ifx\subparagraph\undefined\else
  \let\oldsubparagraph\subparagraph
  \renewcommand{\subparagraph}[1]{\oldsubparagraph{#1}\mbox{}}
\fi
\KOMAoption{captions}{tableheading}
\makeatletter
\makeatother
\makeatletter
\@ifpackageloaded{caption}{}{\usepackage{caption}}
\AtBeginDocument{%
\renewcommand*\contentsname{Table of contents}
\renewcommand*\listfigurename{List of Figures}
\renewcommand*\listtablename{List of Tables}
\renewcommand*\figurename{Figure}
\renewcommand*\tablename{Table}
}
\@ifpackageloaded{float}{}{\usepackage{float}}
\floatstyle{ruled}
\@ifundefined{c@chapter}{\newfloat{codelisting}{h}{lop}}{\newfloat{codelisting}{h}{lop}[chapter]}
\floatname{codelisting}{Listing}
\newcommand*\listoflistings{\listof{codelisting}{List of Listings}}
\makeatother
\makeatletter
\@ifpackageloaded{caption}{}{\usepackage{caption}}
\@ifpackageloaded{subcaption}{}{\usepackage{subcaption}}
\makeatother
\makeatletter
\@ifpackageloaded{tcolorbox}{}{\usepackage[many]{tcolorbox}}
\makeatother
\makeatletter
\@ifundefined{shadecolor}{\definecolor{shadecolor}{rgb}{.97, .97, .97}}
\makeatother
\makeatletter
\makeatother
\ifLuaTeX
  \usepackage{selnolig}  % disable illegal ligatures
\fi

\title{Syllabus - STA101L Summer I 2022}
\author{}
\date{}

\begin{document}
\maketitle

\ifdefined\Shaded\renewenvironment{Shaded}{\begin{tcolorbox}[enhanced, interior hidden, borderline west={3pt}{0pt}{shadecolor}, frame hidden, boxrule=0pt, sharp corners]}{\end{tcolorbox}}\fi

\href{/documents/course-syllabus.pdf}{\textbf{Click here}} \textbf{to
download a PDF copy of the syllabus.}

\hypertarget{course-info}{%
\subsection{Course info}\label{course-info}}

\begin{longtable}[]{@{}
  >{\raggedright\arraybackslash}p{(\columnwidth - 6\tabcolsep) * \real{0.1370}}
  >{\raggedright\arraybackslash}p{(\columnwidth - 6\tabcolsep) * \real{0.2192}}
  >{\raggedright\arraybackslash}p{(\columnwidth - 6\tabcolsep) * \real{0.2603}}
  >{\raggedright\arraybackslash}p{(\columnwidth - 6\tabcolsep) * \real{0.3836}}@{}}
\toprule
\begin{minipage}[b]{\linewidth}\raggedright
\end{minipage} & \begin{minipage}[b]{\linewidth}\raggedright
Day
\end{minipage} & \begin{minipage}[b]{\linewidth}\raggedright
Time
\end{minipage} & \begin{minipage}[b]{\linewidth}\raggedright
Location
\end{minipage} \\
\midrule
\endhead
Lectures & Mon, Tue \& Thu & 3:30 pm - 5:35 pm & Old Chemistry Building
003 \\
Labs & Wed \& Fri & 3:30 pm - 4:45 pm & Old Chemistry Building 003 \\
\bottomrule
\end{longtable}

\hypertarget{overview}{%
\subsection{Overview}\label{overview}}

Welcome to STA101L Data Analysis and Statistical Inference. The goal of
this class is to prepare you to be critical consumers of statistical
analyses in your scientific fields of practice and future professions.
Our point of departure will be to think about data collection: how to
(not) collect data, and how the way in which data are collected impacts
the analysis that we conduct. We will then quickly delve into data
visualization. Ever heard of a mozaic plot or an inter-quartile range?
This is an area of data analysis where creativity and an eye for good
design can make a difference. Once we have good grasp of how to
visualize data, we will construct statistical models to make predictions
and to understand the relations that exist between variables.

While models can be useful (especially if their predictions are
accurate!), all models are inherently \emph{wrong}. The statistical
tests that we will learn in the second half of the course will help us
quantify how (un)certain we are that the models we construct pick up
real patterns in the data and not just background noise.

Throughout the class, you will attend hands-on labs in which you will
learn to implement all these techniques in the statistical computing
software R.

\hypertarget{learning-objectives}{%
\subsection{Learning objectives}\label{learning-objectives}}

By the end of the term you will be able to\ldots{}

\begin{itemize}
\tightlist
\item
  visualize and summarize data sets with numerical and categorical
  variables;
\item
  construct and investigate linear regression models for forecasting;
  this includes fitting, evaluating, comparing and selecting models, as
  well as interpreting their output;
\item
  conduct hypothesis tests and construct confidence intervals for
  proportions, differences between proportions, means, differences
  between means and regression coefficients;
\item
  implement these techniques in R, and use RMarkdown to write
  reproducible reports;
\item
  plan and complete a statistical analysis of a real-world phenomenon
  using visual and numerical summaries, hypothesis tests, confidence
  intervals and regression models.
\end{itemize}

\hypertarget{prerequisite}{%
\subsection{Prerequisite}\label{prerequisite}}

There is no prerequisite for this class.

\hypertarget{community}{%
\subsection{Community}\label{community}}

\hypertarget{duke-community-standard}{%
\subsubsection{Duke Community Standard}\label{duke-community-standard}}

As a student in this course, you have agreed to uphold the
\href{https://studentaffairs.duke.edu/conduct/about-us/duke-community-standard}{Duke
Community Standard} as well as the practices specific to this course.

\hypertarget{inclusive-community}{%
\subsubsection{Inclusive community}\label{inclusive-community}}

It is my intent that students from all backgrounds and perspectives be
well-served by this course, that students' learning needs be addressed
both in and out of class, and that the diversity that the students bring
to this class be viewed as a resource, strength, and benefit. It is my
intent to present materials and organize activities that are respectful
of diversity and in alignment with
\href{https://provost.duke.edu/initiatives/commitment-to-diversity-and-inclusion}{Duke's
Commitment to Diversity and Inclusion}. Your suggestions are encouraged
and appreciated. Please let me know ways to improve the effectiveness of
the course for you personally, or for other students or student groups.

Furthermore, I would like to create a learning environment for my
students that supports a diversity of thoughts, perspectives and
experiences, and honors your identities. To help accomplish this:

\begin{itemize}
\tightlist
\item
  if you feel like your performance in the class is being impacted by
  your experiences outside of class, please don't hesitate to come and
  talk with me; if you prefer to speak with someone outside of the
  course, your academic dean is an excellent resource;
\item
  I (like many people) am still in the process of learning about diverse
  perspectives and identities; if something was said in class (by
  anyone) that made you feel uncomfortable, please let me or a member of
  the teaching team know.
\end{itemize}

\hypertarget{accessibility}{%
\subsubsection{Accessibility}\label{accessibility}}

If there is any portion of the course that is not accessible to you due
to challenges with technology or the course format, please let me know
so we can make appropriate accommodations.

The \href{https://access.duke.edu/students}{Student Disability Access
Office (SDAO)} is available to ensure that students are able to engage
with their courses and related assignments. Students should be in touch
with the Student Disability Access Office to
\href{https://access.duke.edu/requests}{request or update
accommodations} under these circumstances.

\hypertarget{communication}{%
\subsubsection{Communication}\label{communication}}

All lecture notes, assignment instructions, an up-to-date schedule, and
other course materials may be found on the course website at
\href{https://rmorsomme.github.io/website}{rmorsomme.github.io/website}.

I will regularly send course announcements via email, make sure to check
your mail box regularly. If an announcement is sent Monday through
Thursday, I will assume that you have read the announcement by the next
day. If an announcement is sent on a Friday or over the weekend, I will
assume that you have read it by Monday.

\hypertarget{where-to-get-help}{%
\subsubsection{Where to get help}\label{where-to-get-help}}

\begin{itemize}
\tightlist
\item
  If you have a question during lecture or lab, feel free to ask it!
  There are likely other students with the same question, so by asking
  you will create a learning opportunity for everyone.
\item
  The teaching team is here to help you be successful in the course. You
  are encouraged to attend office hours to ask questions about the
  course content and assignments. Many questions are most effectively
  answered as you discuss them with others, so office hours are a
  valuable resource. Please use them!
\item
  Outside of class and office hours, any general questions about course
  content or assignments should be posted on the course forum
  \href{https://sakai.duke.edu/portal/site/37c2d38d-400d-4210-8e42-83e19f9099b3/tool/bab0d8c9-88e8-4138-9490-2c2ad1a33858}{Conversations}.
  There is a chance another student has already asked a similar
  question, so please check the other posts in Conversations before
  adding a new question. If you know the answer to a question posted in
  the discussion forum, I encourage you to respond!
\item
  Emails should be reserved for questions not appropriate for the public
  forum. \textbf{If you email me, please include ``STA 101'' in the
  subject line.} Barring extenuating circumstances, I will respond to
  STA 101 emails within 24 hours Monday - Friday. Response time may be
  slower for emails sent Friday evening - Sunday.
\end{itemize}

Check out the \href{/course-support.html}{Support} page for more
resources.

\hypertarget{textbook}{%
\subsection{Textbook}\label{textbook}}

The class will closely follow the book
\href{https://openintro-ims.netlify.app/}{Introduction to Modern
Statistics} (first edition) by Mine Çetinkaya-Rundel and Johanna Hardin.
This is an open-source book freely available online. You're welcomed to
read on screen or print it out. If you prefer a paperback version you
can buy it at the cost of printing (around \$20) on Amazon. The textbook
store will not carry copies of this text.

Chapters 1-7 of the book \href{https://r4ds.had.co.nz/}{R for Data
Science} by Garret Grolemund and Hadley Wickham (also open-source and
freely available) will also be useful.

\hypertarget{lectures-and-labs}{%
\subsection{Lectures and labs}\label{lectures-and-labs}}

The goal of both the lectures and the labs is for them to be as
interactive as possible. My role as instructor is to introduce you new
tools and techniques, but it is up to you to take them and make use of
them. A lot of what you do in this course will involve writing code, and
coding is a skill that is best learned by doing. Therefore, as much as
possible, you will be working on a variety of tasks and activities
throughout each lecture and lab. You are expected to attend all lecture
and lab sessions and meaningfully contribute to in-class exercises and
discussion.

You are expected to bring a laptop to each class so that you can take
part in the in-class exercises. Please make sure your laptop is fully
charged before you come to class as the number of outlets in the
classroom may not be sufficient to accommodate everyone. More
information on loaner laptops can be found
\href{https://keeplearning.duke.edu/technical-support/}{here}.

\hypertarget{assessment}{%
\subsection{Assessment}\label{assessment}}

Assessment for the course is comprised of three components: regular
homework assignments, a prediction group project and an inference group
project.

\hypertarget{homework-assignments}{%
\subsubsection{Homework assignments}\label{homework-assignments}}

To reduce the number of assignments during the summer session, problem
sets and lab exercises will be merged. In these homework assignments,
you will apply what you've learned during lectures and labs to show
conceptual understanding of the content and complete data analysis tasks
in \texttt{R}. You may discuss homework assignments with other students;
however, homework should be completed and submitted individually.
Homework must be typed up using RMarkdown and submitted as a PDF on
\href{https://www.gradescope.com/courses/394638}{Gradescope}.

\emph{The homework assignment with the lowest grade will be dropped at
the end of the term.}

\hypertarget{projects}{%
\subsubsection{Projects}\label{projects}}

There will be two group projects. Through these, you have the
opportunity to demonstrate what you've learned in the course thus far.
The projects will focus on the two pillars of statistics and data
science: \emph{prediction} and \emph{inference}. In the first project,
you will construct a regression model for prediction. The goal will be
to build a model that provides predictions that are as accurate as
possible. In the second project, you will analyze a phenomenon that
interests you using real-world data. More detail about the projects will
be given during the term.

\hypertarget{grading}{%
\subsection{Grading}\label{grading}}

The final course grade will be calculated as follows:

\begin{longtable}[]{@{}ll@{}}
\toprule
Assessment & Percentage \\
\midrule
\endhead
Homework & 50\% \\
Prediction project & 20\% \\
Inference project & 30\% \\
\bottomrule
\end{longtable}

The final letter grade will be determined based on the following
thresholds:

\begin{longtable}[]{@{}ll@{}}
\toprule
Letter Grade & Final Course Grade \\
\midrule
\endhead
A & \textgreater= 93 \\
A- & 90 - 92.99 \\
B+ & 87 - 89.99 \\
B & 83 - 86.99 \\
B- & 80 - 82.99 \\
C+ & 77 - 79.99 \\
C & 73 - 76.99 \\
C- & 70 - 72.99 \\
D+ & 67 - 69.99 \\
D & 63 - 66.99 \\
D- & 60 - 62.99 \\
F & \textless{} 60 \\
\bottomrule
\end{longtable}

Note that the final grade may be curved.

\hypertarget{five-tips-for-success}{%
\subsection{Five tips for success}\label{five-tips-for-success}}

Your success on this course depends very much on you and the effort you
put into it. The course has been organized so that the burden of
learning is on you. Your TA and I will help you by providing you with
materials and answering questions, but for this to work you need to do
the following:

\begin{enumerate}
\def\labelenumi{\arabic{enumi}.}
\tightlist
\item
  complete the assigned reading before the lectures;
\item
  ask questions \emph{quickly}; don't let a day pass by with lingering
  questions;
\item
  do the homework assignments thoroughly;
\item
  practice, practice, practice;
\item
  don't procrastinate; start on the homework assignments and the
  projects early.
\end{enumerate}

\hypertarget{course-policies}{%
\subsection{Course policies}\label{course-policies}}

\hypertarget{academic-integrity}{%
\subsubsection{Academic integrity}\label{academic-integrity}}

\textbf{Don't cheat!}

All students must adhere to the
\href{https://trinity.duke.edu/undergraduate/academic-policies/community-standard-student-conduct}{Duke
Community Standard (DCS)}: Duke University is a community dedicated to
scholarship, leadership, and service and to the principles of honesty,
fairness, and accountability. Citizens of this community commit to
reflect upon these principles in all academic and non-academic
endeavors, and to protect and promote a culture of integrity.

To uphold the Duke Community Standard:

Students affirm their commitment to uphold the values of the Duke
University community by signing a pledge that states:

\begin{itemize}
\tightlist
\item
  I will not lie, cheat, or steal in my academic endeavors;
\item
  I will conduct myself honorably in all my endeavors;
\item
  I will act if the Standard is compromised
\end{itemize}

Regardless of course delivery format, it is your responsibility to
understand and follow Duke policies regarding academic integrity,
including doing one's own work, following proper citation of sources,
and adhering to guidance around group work projects. Ignoring these
requirements is a violation of the Duke Community Standard. If you have
any questions about how to follow these requirements, please contact
Jeanna McCullers
(\href{mailto:jeanna.mccullers@duke.edu}{\nolinkurl{jeanna.mccullers@duke.edu}}),
Director of the Office of Student Conduct.

\hypertarget{collaboration-policy}{%
\subsubsection{Collaboration policy}\label{collaboration-policy}}

Only work that is clearly assigned as team work should be completed
collaboratively.

\begin{itemize}
\tightlist
\item
  The homework assignments must be completed individually and you are
  welcomed to discuss the assignment with classmates at a high level
  (e.g., discuss what's the best way for approaching a problem, what
  functions are useful for accomplishing a particular task, etc.).
  However you may not directly share answers to homework questions
  (including any code) with anyone other than myself and the TA.
\item
  For the projects, collaboration between teams at a high level is also
  allowed; however, you may not share code or components of the project
  with other teams.
\end{itemize}

\hypertarget{policy-on-sharing-and-reusing-code}{%
\subsubsection{Policy on sharing and reusing
code}\label{policy-on-sharing-and-reusing-code}}

I am well aware that a huge volume of code is available on the web to
solve any number of problems. You may make use of any online resources
(e.g.~RStudio Community, StackOverflow) but you must explicitly cite
where you obtained any code you directly use (or use as inspiration).
Any recycled code that is discovered and is not explicitly cited will be
treated as plagiarism. On homework assignments you may not directly
share code with another student in this class, and on the projects you
may not directly share code with another team.

\hypertarget{late-work-policy}{%
\subsubsection{Late work policy}\label{late-work-policy}}

The due dates for assignments are there to help you keep up with the
course material and to ensure the teaching team can provide feedback
within a timely manner. Given the fast pace of a summer class,
\textbf{any assignment submitted after the deadline will not be graded}.
The homework assignments will be published early to give you ample time
to work on them. Note that the lowest homework will be dropped.

\hypertarget{waiver-for-extenuating-circumstances}{%
\subsubsection{Waiver for extenuating
circumstances}\label{waiver-for-extenuating-circumstances}}

If there are circumstances that prevent you from completing a lab or
homework assignment by the stated due date, you may email
\href{mailto:raphael.morsomme@duke.edu}{Raphael Morsomme} \textbf{and}
our TA \href{mailto:rohit.roy@duke.edu}{Rohit Roy} before the deadline
to obtain a 24-hour extension. In your email, you only need to request
the waiver; you do not need to provide explanation. This waiver may only
be used for once in the semester, so only use it for a truly extenuating
circumstance.

If there are circumstances that are having a longer-term impact on your
academic performance, please let your academic dean know, as they can be
a resource. Please let a member of the instruction team know if you need
help contacting your academic dean.

\hypertarget{regrade-request-policy}{%
\subsubsection{Regrade request policy}\label{regrade-request-policy}}

Regrade requests must be submitted on
\href{https://www.gradescope.com/courses/394638}{Gradescope} within 48
hours of when an assignment is returned. Regrade requests will be
considered if there was an error in the grade calculation or if you feel
a correct answer was mistakenly marked as incorrect. Requests to dispute
the number of points deducted for an incorrect response will not be
considered. Note that by submitting a regrade request, the entire
question will be graded which could potentially result in losing points.

\emph{No grades will be changed after the inference project report is
due.}

\hypertarget{attendance-policy}{%
\subsubsection{Attendance policy}\label{attendance-policy}}

Responsibility for class attendance rests with individual students.
Since regular and punctual class attendance is expected, students must
accept the consequences of failure to attend. More details on Trinity
attendance policies are available
\href{https://trinity.duke.edu/undergraduate/academic-policies/class-attendance-and-missed-work}{here}.

However, there may be many reasons why you cannot be in class on a given
day, particularly with possible extra personal and academic stress and
health concerns this term. Lab time is dedicated to working on your lab
assignments and collaborating with your teammates on your project. If
you miss a lab session, make sure to communicate with your team about
how you can make up your contribution. If you know you're going to miss
a lab session and you're feeling well enough to do so, notify your
teammates ahead of time. Overall these policies are put in place to
ensure communication between team members, respect for each others'
time, and also to give you a safety net in the case of illness or other
reasons that keep you away from attending class.

\hypertarget{attendance-policy-related-to-covid-symptoms-exposure-or-infection}{%
\subsubsection{Attendance policy related to COVID symptoms, exposure, or
infection}\label{attendance-policy-related-to-covid-symptoms-exposure-or-infection}}

Student health, safety, and well-being are the university's top
priorities. To help ensure your well-being and the well-being of those
around you, please do not come to class if you have symptoms related to
COVID-19, have had a known exposure to COVID-19, or have tested positive
for COVID-19. If any of these situations apply to you, you must follow
university guidance related to the ongoing COVID-19 pandemic and current
health and safety protocols. If you are experiencing any COVID-19
symptoms, contact student health at 919-681-9355. To keep the university
community as safe and healthy as possible, you will be expected to
follow these guidelines. Please reach out to me and your academic dean
as soon as possible if you need to quarantine or isolate so that we can
discuss arrangements for your continued participation in class.

Please read and follow university guidance
\href{https://coronavirus.duke.edu/}{here}. The current guidelines are
for students and instructors to \textbf{wear masks during class}.

\hypertarget{inclement-weather-policy}{%
\subsubsection{Inclement weather
policy}\label{inclement-weather-policy}}

In the event of inclement weather or other connectivity-related events
that prohibit class attendance, I will notify you how we will make up
missed course content and work. This might entail holding the class on
Zoom synchronously or watching a recording of the class.

\hypertarget{policy-on-video-recording-course-content}{%
\subsubsection{Policy on video recording course
content}\label{policy-on-video-recording-course-content}}

If you feel that you need record the lectures, you must get permission
from the instructor ahead of time and these recordings should be used
for personal study only, no for distribution. The full policy on
recording of lectures falls under the Duke University Policy on
Intellectual Property Rights, available at
\href{https://provost.duke.edu/sites/default/files/FHB_App_P.pdf}{provost.duke.edu/sites/default/files/FHB\_App\_P.pdf}.
Unauthorized distribution is a cause for disciplinary action by the
Judicial Board.

\hypertarget{learning-during-a-pandemic}{%
\subsection{\texorpdfstring{\textbf{Learning during a
pandemic}}{Learning during a pandemic}}\label{learning-during-a-pandemic}}

I want to make sure that you learn everything you were hoping to learn
from this class. If this requires flexibility, please don't hesitate to
ask.

\begin{itemize}
\item
  You \emph{never} owe me personal information about your health (mental
  or physical) but you're always welcome to talk to me. If I can't help,
  I likely know someone who can.
\item
  I want you to learn lots of things from this class, but I primarily
  want you to stay healthy, balanced, and grounded during this health
  crisis.
\end{itemize}

\emph{Note:} If you've read this far in the syllabus, email me a picture
of your pet if you have one or your favourite meme!

\hypertarget{important-dates}{%
\subsection{Important dates}\label{important-dates}}

\begin{itemize}
\tightlist
\item
  \textbf{May 11:} Classes begin (Monday meeting schedule)
\item
  \textbf{May 13:} Drop/add ends
\item
  \textbf{May 16:} Regular class meeting schedule begins
\item
  \textbf{May 30:} Memorial Day holiday, no class is held
\item
  \textbf{May 31:} Prediction project
\item
  \textbf{June 8:} Last day to withdraw with W
\item
  \textbf{June 16}: Inference project presentation
\item
  \textbf{June 17}: Classes end
\item
  \textbf{June 20:} Juneteenth holiday
\item
  \textbf{June 21}: Reading period
\item
  \textbf{June 23}: Last assignment: inference project report
\end{itemize}

Click
\href{https://registrar.duke.edu/sumemr-2022-academic-calendar}{here}
for the full Duke academic calendar.

\hypertarget{attribution}{%
\subsection{Attribution}\label{attribution}}

Some portions of this syllabus are based on the syllabus of
\href{https://sta210-s22.github.io/website/}{STA 210 - Spring 2022} by
Prof.~Mine Çetinkaya-Rundel.

\end{document}
